\documentclass[a4paper]{article}

\usepackage[a4paper,left=3cm,right=2cm,top=2.5cm,bottom=2.5cm]{geometry}
\usepackage[utf8]{inputenc}
\usepackage[T1]{fontenc}
\usepackage{textcomp}
\usepackage{amsmath, amssymb, amsthm}
\usepackage{hyperref}
\usepackage{booktabs}



% figure support
\usepackage{import}
\usepackage{xifthen}
\pdfminorversion=7
\usepackage{pdfpages}
\usepackage{transparent}
\newcommand{\incfig}[1]{%
    \def\svgwidth{\columnwidth}
    \import{./figures/}{#1.pdf_tex}
}

\pdfsuppresswarningpagegroup=1
\title{\vspace{-2cm}INF224 | TP README}
\author{Cheng-Yen Wu}
\date{2022}

\begin{document}
\maketitle

\section{Démarrage}
\section{Classe de base}
\section{Programme de test}
\section{Photos et videos}

\textbf{Comment appelle-t'on ce type de méthode et comment faut-il les déclarer
?}
\textbf{ Si vous avez fait ce qui précède comme demandé, il ne sera plus
possible d'instancer des objets de la classe de base. Pourquoi ? }

\section{Traitement uniforme}

\textbf{Quelle est la propriété caractéristique de l'orienté objet qui permet de
faire cela ?}

\textbf{Qu'est-il spécifiquement nécessaire de faire dans le cas du C++ ?} 

\textbf{Quel est le type des éléments du tableau : le tableau doit-il contenir
des objets ou des pointeurs vers ces objets ? Pourquoi ?  Comparer à Java.} 

\section{Films et tableaux}
\section{Destruction et copie des objets}

Parmi les classes précédemment écrites quelles sont celles qu'il faut modifier
afin qu'il n'y ait pas de fuite mémoire quand on détruit les objets ? Modifiez
le code de manière à l'éviter.



La copie d'objet peut également poser problème quand ils ont des variables
d'instance qui sont des pointeurs. Quel est le problème et quelles sont les
solutions ? Implementez-en une. 

\section{Créer des groupes}
Le groupe ne doit pas détruire les objets quand il est détruit car un objet
peut appartenir à plusieurs groupes (on verra ce point à la question suivante).
On rappelle aussi que la liste d'objets doit en fait être une liste de pointeurs
d'objets. Pourquoi ? Comparer à Java. 


\section{Gestion automatique de la mémoire}
Comment peut-on l'interdire, afin que seule la classe servant à manipuler les
objets puisse en créer de nouveaux ? 
\section{Gestion cohérente des données}
\section{Client/serveur}
\section{Sérialisation/désérialisation}
\section{Traitement des erreurs}






    

\bibliography{referencej}
\bibliographystyle{plain}
\end{document}


